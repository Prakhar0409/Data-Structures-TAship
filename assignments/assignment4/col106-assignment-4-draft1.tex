\documentclass[20pt]{article}
\usepackage[inner=3cm,outer=3cm,top=2.5cm,bottom=2.5cm]{geometry}
\title{COL106: Assignment 4}
\author{Team Prakhar}
\date{}
\newcommand{\homework}[5]{
   \pagestyle{myheadings}
   \thispagestyle{plain}
   \newpage
   \setcounter{page}{1}
   \noindent
   \begin{center}
   \framebox{
      \vbox{\vspace{2mm}
    \hbox to 5.5in { {\bf COL106:~Data Structures and Algorithms\hfill} }
       \vspace{6mm}
       \hbox to 5.5in { {\Large \hfill #1  \hfill} }
       \vspace{6mm}
       \hbox to 5.5in { {\it Instructor: #3 \hfill #2}}
       %\hbox to 6.28in { {\it TA: #4  \hfill #5}}
      \vspace{2mm}}
   }
   \end{center}
   \markboth{#5 -- #1}{#5 -- #1}
   \vspace*{4mm}
}
\begin{document}
\homework{Assignment 4}{Deadline: TBA 11:55pm}{Prof. Saroj Kaushik}{}{}

% \maketitle

\section{Statement}

Amravathi, is a newly established township on the banks of river Krishna in Andhra Pradesh. The government is looking forward to strengthening the road network of the city in order to boost the development. At an initial stage, the government has decided a few prime junctions/points in the city (each junction has a value associated - lets call it \textbf{traffic\_light\_time}). The chief engineer has presented a list of roads to be built in the future and the time required for building each road (\textbf{build\_time}: an integer representing number of months required) as well as the time required to move on the road from one end to another once the road is complete (\textbf{traverse\_time}: an integer representing minutes required to go across the road).
\\\\
In order to speed up things, we wish to connect all the identified prime junctions/points in the least possible time (assume work on roads cannot be simultaneous i.e. only work on one road can progress at any given time). This reduces problem to identification of the minimum spanning tree in the network of roads (assume traffic signals to be not functional yet). Note that this operation (of finding the MST) might be needed even after initial construction of the roads while repairing the roads.
\\\\
Soon, all the roads in the original plan (irrespective of whether they were in the MST or not) are built. The government now wishes to develop a travel app for the citizens to ease their movement within the city. Any person wanting to travel by road, enters a source and a destination and hopes that the app returns the quickest way to reach the destination. This eventually boils down to the following: 
\\\\
You want to go from one node to another in the shortest amount of time. Time required to traverse for edge (road) $e$ is $e.traverse\_time$ time units. At any junction $x$ having the traffic signal value $x.traffic\_light\_time$, going through that node will force you to wait until the time spent is divisible by $x.traffic\_light\_time$.\\
%You now wish to have a unique enumeration of the prime junctions given a start junction. For this we use a simple DFS with the strategy that at any new node we visit, visit its neighbours in descending order of their ids.


\textbf{NOTE: You are allowed (\& encouraged) to use the priority\_queue from the Standard Template Library (STL)} (Hint: STL priority queue does not support updation outright. However, it is possible to still use it for implementation of Dijkstra's algorithm with a very slight change in time and space complexity)
\\\\
You thus need to support the following operations:
\begin{itemize}

\item \textbf{add-junction} $<$x$>$$<$y$>$: Add a junction $x$ with $traffic\_light\_time = y$.
\item \textbf{add-road} $<$i$>$ $<$j$>$ $<$m$>$ $<$n$>$: Add a road between junction $i$ and $j$ with $build\_time = m$ and $traverse\_time = n$.
\item \textbf{demolish-road} $<$i$>$ $<$j$>$: Demolish the road between the prime point $i$ and $j$. 
\item \textbf{print-mst}: Print the roads (represented as $(x \ y)$ where $x<y$ and the road connects junctions $x$ and $y$) in the minimum spanning tree of the current junction-road graph ordered such that if a road $(x_1 \ y_1)$ is placed before $(x_2 \ y_2)$, then either $x_1 < x_2$ or if $x_1 = x_2$ then $y_1 < y_2$. If multiple MSTs exist, then output any one of them. \textbf{You must use STL for sorting}. If there is tree spanning all vertices, just output -1.
\item \textbf{dfs-traverse} $<$i$>$ $<$k$>$: Visit all junctions connected to $i$ in a Depth-First-Search fashion starting at $i$. \textbf{Visit neighbours of each junction in descending order of their junction-ids}. Print the last $k$ junctions in the traversal in the order you visited them (might include $i$ itself). If there are $k1$ no. of junctions visited in total such that $k1 < k$, then print those $k1$ junctions.
\item \textbf{quick-travel} $<$i$>$ $<$j$>$: Print the minimum time needed to reach junction $i$ starting at $j$ at time 0. Also, output the junction ids in order of your suggested visit. If multipltiple such paths with the same travel time exist, then you are free to print any one of them. If it is not possible to reach $i$ from $j$, then simply print -1.
\end{itemize}


\section{Input Format}
\begin{itemize}
\item All IO operations are to be done using standard input/output.
\item The first line of the input contains two space separated integers $V$ - the no. of prime junctions/points and $E$ - the no. of proposed roads.
\item Following $V$ lines contains 2 space separated integers - $junction-id$ and corresponding $traffic\_light\_time$ - representing the $V$ prime junctions.
\item Following $E$ lines contains 4 space separated integers - $junction-id-1$, $junction-id-2$, $build\_time$  and $traverse\_time$  corresponding to the $E$ proposed roads.
\item Next line has a single integer $Q$ - the no. of operations to be performed.
\item $Q$ lines follow, each consisting of 1 operation in the following format: $<$operation-code$>$ $[$params$]$.
\end{itemize}

\begin{center}
\begin{tabular}{|c|c|}
\hline 
\textbf{Code} & \textbf{Operation} \\
\hline 
1 & add-junction \\ 
\hline 
2 & add-road \\ 
\hline 
3 & demolish-road \\ 
\hline 
4 & print-mst \\ 
\hline 
5 & dfs-traverse \\ 
\hline 
6 & quick-travel \\ 
\hline 
\end{tabular} 
\end{center}
\section{Output Format}
\begin{itemize}
\item Operations 1, 2 and 3 have no outputs
\item Operation 4 outputs the roads (one road per line) to keep in the MST in sorted order as described in the operation description, where each edge is represented by space separated junction-ids in ascending order
\item Operation 5 must output, in a single line, the space separated junction-ids in the order of DFS traversal
\item Operation 6 must output, in a single line, the minimum time, along with the junction-ids in expected order of visit (including the source and the destination junction ids), separated by spaces
\end{itemize}

\section{Limits}
\begin{itemize}
\item $0 \leq V, E \leq 10^6$
\item $1 \leq traverse\_time, build\_time, traffic\_light\_time \leq 10^9$
\item For every junction, the $0 \leq junction-id \leq 10^6$
\item At any time there is atmost 1 road between any 2 junctions
\end{itemize}

%\section{Compilation}
%Your program will be compiled using the following command:
%\[\]

\section{Sample IO}
\subsection{Input}
\hspace{10mm} 4 4\\
\-\hspace{10mm} 2 7\\
\-\hspace{10mm} 3 5\\
\-\hspace{10mm} 1 3\\
\-\hspace{10mm} 4 4\\
\-\hspace{10mm} 2 3 10 2\\
\-\hspace{10mm} 4 2 1 1\\
\-\hspace{10mm} 4 3 3 2\\
\-\hspace{10mm} 1 3 3 2\\
\-\hspace{10mm} 8\\
\-\hspace{10mm} 4\\
\-\hspace{10mm} 1 5 10\\
\-\hspace{10mm} 2 5 3 3 10\\
\-\hspace{10mm} 1 10 4\\
\-\hspace{10mm} 2 10 1 3 2\\
\-\hspace{10mm} 2 4 10 2 2\\
\-\hspace{10mm} 5 5 5\\
\-\hspace{10mm} 6 10 2\\

\subsection{Output}
\-\hspace{10mm} 1 3\\
\-\hspace{10mm} 2 4\\
\-\hspace{10mm} 3 4\\
\-\hspace{10mm} 3 4 10 1 2\\
\-\hspace{10mm} 7 10 1 3 2\\

\section{Submission Instructions}
\begin{itemize}
\item You are required strictly follow the instructions given below.
\item You must submit a single program file which must be named with your \textit{KerberosID.cpp} in small case. For example, if your kerberos id is ’ch1150999’ then the file name should be ch1150999.cpp. WARNING: Do not submit any other file formats just renaming them to cpp.
\item \textbf{You will be penalized if your submission does not conform to this requirement}.
\end{itemize}

\section{Other Instructions}
\begin{itemize}
\item The assignment is to be done individually.
\item All submissions must use C/C++ for the programming assignment.
\item You are \textbf{encouraged} to use the standard template library (STL) to ease out your implementation of the assignment. It will be useful in almost all your future projects that require you to write C++ code.
% \item This assignment has been designed to enforce use of linked lists over arrays so the limits are very high. Remember that you would not be able to allocate memory of sizes O($10^9$). So you will have to be smart with your memory allocations and avoid any unnecessary memory usages.
\item You are not allowed to discuss or take help from anybody outside the class. You may only discuss but are not allowed to share code with anyone inside the class.
\item We will run plagiarism detection on your submissions. People found guilty will be penalised as per the instructions mentioned in the beginning of the course
\end{itemize}

% \section{Additional Help}
% \begin{itemize}
% \item Be sure that your implementation is correct. You may want to use debuggers like gdb to examine the state of your program at various points during the execution.
% \item Be careful of memory leaks. Check out tools like valgrind in case you are unsure. Memory leaks are particularly serious issues and critical for programs like daemons (programs that keep running in the background for long periods of time), and search engine core is an example of this. So you will be penalised for any leaking memory.
% \end{itemize}

\end{document}
